\def\format{complete}
\part{Characterizing functional discriminability across the brain using large-scale classification}
\label{characterizingfunctionaldiscriminabilityacrossthebrainusinglarge-scaleclassification}

by Alejandro de la Vega

\chapter{Introduction}
\label{introduction}

Mapping cognitive functions to brain anatomy is central to understanding the functional organization of the human brain. To this end, many functional magnetic resonance imaging (fMRI) studies scan participants while they perform a task purported to elicit a specific process in order to determine which regions activate. This enterprise has led to rapid advances in neuroscience with the discovery of areas that reliably activate for specific processes, such as the fusiform face area for face perception. However, the information that can be gleaned from any one study is limited by the scope of cognitive processes studied and noise inherent to brain measurement. This is especially problematic given that there is likely a many-to-many mapping between functions and regions, such that regions perform multiple functions and functions activate multiple regions. 

Fortunately, the recent development of large-scale neuroimaging databases, such as Neurosynth and BrainMap, have enabled a new breed of approaches that harness the power of many studies to make more robust and nuanced inferences about the functional properties of the brain. These databases have been used for a variety of meta-analytic techniques, including generating activity maps of various cognitive functions and organizing regions into functional networks. They have also been used to characterize the functional specialization of specific areas of the brain, such as the insula (Chang et al.,), as well as the diversity of function across the entire brain (Anderson et al., ). These meta-analytic quantifications of brain function are particularly useful due to the breadth of task activations being analyzed. In doing so, we are able to discern the specificity and variety of function to brain mappings. 

While these descriptors are useful for knowing what the most associated functions for each region are, it is unknown how useful these functions are in predicting activity in a region. [THIS THIS INTO PREVIOUS SENTENCE]
That is, while activity in a region may be correlated to a set of functions, it is unclear if the presence of a function in a study is consistently predictive of activity in that region. Knowing how well we can discriminate regions allows us to know how robust these functional descriptors are for different region. Presumably, regions that have strong associations with specific cognitive functions, such as visual cortex with ‘vision’, should be easy to discriminate while regions with less consistent mappings between function and structure, such as frontal pole, may be more difficult. Moreover, regions with different functional charachteristics may different in the number of cognitive functions necessary to predict their activity. For a region with a fairly specific functon, such as auditory cortex, it may be sufficient to know a paper studied auditory processes to reliability predict activity in that region. Alternatively, regions with broader, less identifiable functions, such as prefrontal cortex, may require a greater number of functional descriptors to predict activation. 

In the present study, we quantified discrimiability across the brain by using machine-learning classifiers to predict which studies in Neurosynth showed activity in individual regions. First, we divided the brain into discrete regions by clustering voxels using coactivation patterns. Next, for each region, we trained a penalized linear classifier to discriminate studies that activated that region versus those that did not. As there is no consensus cognitive ontology, we used as training features 80 topics derived from word co-occurances that map well onto established activation patterns for different cognitive phenomena (Poldrack et al.,). Finally, we tested our classifiers’ on unseen data using cross-validation to yield each region’s discrimiability as well as the importance of each topic in predicting activity in that region.

[Add section here about pairwise results][Perhaps discuss some additional advantages of my approach here]

\chapter{Methods}
\label{methods}

We analyzed the Neurosynth database (neurosynth.org), a repository 8,000 (give specific number) neuroimaging studies and XXXXX activations. Each observation in the database contains the peak activations for all contrasts reported in a study’s table as well as the frequency of all of the words in the body of the text. Although meta-analyses of cognitively relevant words yield activation maps consistent with neuroscienific knowledge, there is a high amount of redundancy between words (e.g. ‘memory’ and ‘episodic’) and potential ambiguity as to the meaning of an indvidual word out of context. Thus, we employed a set of 80 topics derived using latent dirichlet allocation topic-modeling (see Poldrack et al., 2012). The resulting topics reflect the underlying semantic structure of neuroimaging studies while simultaneously yielding more robust meta-analytic maps for individual features. Each study in the database has a load weight to each topic reflecting the semantics of the text. 

To estimate discriminability for each region, we trained a classifier to discriminate between studies that activated that region versus studies that did not. We defined ‘active’ studies as those that activated at least 50 voxels in the ROI and ‘inactive’ as those that did not activate any voxels. Active studies were labeled as the positive class (coded as 1) and inactive studies were labeled as the negative class (coded as 0). Next, we trained a regularized linear regression with an l2-penalty, or ridge regression, to predict the label of each study in a region’s set. Ridge regression is well suited for classification with many co-linear features; large coefficients are penalized, resulting in a dense yet stable solution with class-leading predictive performance. We treated the regression as a binary classification problem by converting predictions for unseen data above 0.5 to 1 (‘positive’) and below 0.5 to 0 (‘negaitve’). To avoid overfitting, we tested our models using 4-fold cross validation. Each model was on trained 3\slash 4ths of the data and asked to predict the class of the remaining 1\slash 4th, circulating over the dataset such that the model was trained and tested on the entire dataset. The mean score across the four folds was used as the final measure of discriminability for each region. We scored the classifiers using by taking area under the curve of the receiver operating charachteristic (auc-roc). This common classification metric, ranging from 0.5 for chance to 1 for flawless predictions, is not biased by the balance of sample sizes between the two classes, which can vary between regions. Finally, for each region, we extracted the regression weights for each topic (‘topic weights’); positive weights indicate the discussion of that topic in a study predict activity in the region, while negative weights indicate the topic is anti-predictive of activity. 

We estimated descriminability for four parcellations of the brain that varied in resolution. We defined regions by clustering voxels together with similar coactivation profiles across studies in the database. We used Ward clustering to group voxels into 10, 20, 30, 40, 50 and 60 clusters. Consistent with the topology of the human brain, clusters consisted of contigous voxels or bilateral structures.

\chapter{Results}
\label{results}

\section{Discriminability across the brain}
\label{discriminabilityacrossthebrain}

Mean discriminability across the four parcellations was above chance (mean auc-roc = 0.57) and peaked when using 20 regions (Figure 1). The low accuracy using 10 regions suggesst that at this resolution regions are too heteogenous to be accurately discriminated, supported by the fact that the regions with most voxels in that parcellation were the most difficiult to discriminate. However, increasing the resolution past 20 regions led to a slow decrease in discriminability. Less studies activated each region as the resolution increased, ranging from a mean of 4858 studies per region using 10 parcels to 1706 with 60 regions, suggesting that there was insufficient data to support training at high resolutions. Henceforth, we focus on the results using 40 regions as they strike a balance between resolution and performance.

Regional discriminability showed wide variability across regions, ranging from near chance, for motor cortex, performance to robust classification, for the superior parietal lobule (Figure 2, auc-roc range: 0.51 - 0.65). Discriminability was slightly lower for larger regions (r = 0.12), but this effect is sufficiently small to allow for comparison across parcels of different size (see Supplemental Materials). 

Regions also varied in the topics that were predictive of their activation. In Figure 3, we plot the topic weights for each region, arranged by the region’s discriminability. Predictive topics are consistent with existing knowldge about function to anatomy mappings. For example, for auditory cortex, a region that was highly discriminable, the most predictive topic was most associated with the words ’speech, auditory, and sounds’. Likewise, bilateral amygdala, a moderately discrimable region, was most predicted by a topic most associated with the word ‘emotional’. Even regions with low discriminabilty, such as right middle frontal gyrus, was most predicted by a topic about ‘inhibition, conflict, and control’, consistent with the role of lateral prefrontal cortex in cognitive control.

While most regions seem to be most predicted by a topic related to its known cognitive function, we asked if the discriminability of a region affected the strength or distribution of the topic weights. In other words, do more discriminable regions have stronger relationships to specific cognitive phenomena? Discriminability did not correlate with overall topic weights, but did correlate with the absolute value of the weights. This suggests more discriminable regions are more strongly predicted by certain topics, but also strongly antipredicted by others. For example, the parietal lobule is easily discriminable and activity within it is strongly anti predicted by the topic described the words ‘wm’ , ‘integrity’, ‘fractional’.

[Multivariateness analysis]

[Then introduce pairwise analysis] 
\end{document}
